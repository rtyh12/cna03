\documentclass {article}
\usepackage{amsmath}
\setlength{\parindent}{0cm}

\usepackage{hyperref}
\usepackage{graphicx}
\usepackage{subcaption}
\usepackage[margin=1.5in]{geometry}

\usepackage{fancyhdr}
\pagestyle{fancy}
\lhead{\textbf{Complex Network Analysis} \\ Assignment 3\\}
\rhead{Maria Kagkeli \\ Maria Regina Lily \\ Mihai Verzan}
\headheight 10pc
\voffset -10pc

\begin{document}

\section{Clustering Coefficients}
\subsection{}

$ k_{A, B} = \ell + 1$, $k_{\ell} = 2 $, so:
\begin{align*}
\langle C \rangle & = \frac{ 1 }{ N } \sum\limits_{ i=1 }^N \frac{ 2 L_i }{ k_i (k_i - 1) } \\ \\
& = \frac{ 1 }{ N } \left( \underbrace{\sum\limits_{ A,B } \frac{ 2 \ell }{ (\ell + 1) \ell }}_\text{nodes $A$, $B$} + \underbrace{ \sum\limits_{ i=0 }^{N-2} \frac{ 2 \cdot 1 }{ 2 \cdot 1 }}_\text{nodes $l$} \right) \\ \\
& = \frac{ 1 }{ N } \left( \frac{ 4 }{ \ell + 1 } + N - 2  \right) \\ \\
& = \frac{ 1 }{ N } \left( \frac{ 4 }{ N - 1 } + N - 2  \right) \\ \\
& = \frac{ 1 }{ N } \left( \frac{ 4 + (N - 2)(N - 1) }{ N - 1 } \right) \\ \\
& = \frac{ 1 }{ N } \left( \frac{ 4 + N^2 - 3N + 2 }{ N - 1 }  \right) \\ \\
& = \frac{ N^2 - 3N + 6 }{ N^2 - N }. \\ \\
\end{align*}

\subsection{}
\text{According to hint:}
\begin{align*}
C_{\Delta} &= \frac{ 3 \cdot \# triangles }{ \# triples },
& \# triples = \sum\limits_{ i=1 }^N \frac{ k_i (k_i - 1) }{ 2 }.\\
\end{align*}

In the given graph, $ \# $ triangles = $ \ell $, therefore:
$$ C_{\Delta} = \frac{ 3 \ell }{ \sum\limits_{ i=1 }^N \frac{ k_i (k_i - 1) }{ 2 }}. $$
 
From problem 1.1: $ k_{A, B} = \ell + 1$,    $k_{\ell} = 2 $, so:
\begin{align*}
 C_{\Delta} &=   \frac{ 3 \ell }{ \sum\limits_{ A,B} \frac{ (\ell + 1) \ell }{ 2 } + \sum\limits_{ i=0 }^{N-2} \frac{ 2 \cdot 1 }{ 2 }} \\ \\
& =\frac{ 3 \ell }{ \ell (\ell + 1) + N - 2 } \\ \\
& = \frac{ 3 (N - 2) }{ (N-1)(N-2) + N - 2 } \\ \\
& = \frac{ 3 (N - 2) }{ N^2 - 3N + 2 + N - 2 } \\ \\
& = \frac{ 3 (N - 2) }{ N^2 - 2N } \\ \\
& = \frac{ 3 (N - 2) }{ N (N - 2) } \\ \\
& = \frac{ 3 }{ N }.
\end{align*}


\subsection{}
As $ N \to \infty $, the average clustering coefficient $ \langle C \rangle = (\frac{ N^2 - 3N + 6 }{ N^2 - N }) $ goes to $ 1 $. These reflects the fact that nodes $ A $ and $ B $ are connected to all others, and as the number of nodes goes to infinity, these connections dominate the coefficient. The global clustering coefficient $ C_{\Delta} = \frac{ 3 }{ N } $ goes to 0, meaning that the number of triangles grows much more slowly than the number of triples. These limits show that in some cases, there is a very clear difference between the two formulations. 

\newpage

\end{document}